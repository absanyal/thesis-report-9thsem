\chapter{\label{intro}Introduction}

\setcounter{equation}{0}
\setcounter{table}{0}
\setcounter{figure}{0}
%\baselineskip 20pt

\section{Green's Functions}
	The Green's function is an operator used to find the eigenvalues of a  Hamiltonian, $ \hat{\mathcal{H}} $, defined as
	\begin{align}
		\hat{G}\left(\omega\right) = \left[ \left( \omega + \I \eta\right) \mathbb{I} - \ham \right]^{-1}
	\end{align}
	where $ \omega $ is frequency and $ \eta $ is a small, non-zero constant called the regulator. $ \mathbb{I} $ is an identity matrix of the same dimension as $ \ham $.
	
	When operated on an eigenstate state $ \ket{\tilde{\psi_i}} $, we get the eigenvalue of $ \hat{G} $ as
	\begin{align}
		G\left(\omega\right) = \left[ \omega + \I \eta - \mathcal{E}_i \right]^{-1}
	\end{align}
	where $ \mathcal{E}_i $ is the eigenvalue of $ \ham $ with respect to $ \ket{\tilde{\psi_i}} $. Thus, we see that the eigenvalue of the Green's function is a function of $ \omega $ with poles at the eigenvalues of the Hamiltonian. The regulator prevents the eigenvalue from blowing up at these points. While methods to exactly diagonalize the Hamiltonian, such as \textit{Lanczos algorithm}, do exist, the Green's function method has several advantages over exact diagonalization methods. For example, exact diagonalization forces us to calculate the entire spectrum at a time. On the other hand, calculating the Green's function allows us to find the eigenvalues within any interval of $\omega$.
    
\section{Calculating physical observables from Green's function}
	Given a many body system, we can construct a many body basis consisting of the states of the system.
	\begin{align}
		\mathcal{B} = \{ \ket{\psi_1}, \ket{\psi_2}, \dots \}
	\end{align} 
	
	If we know how $ \ham $ acts on elements of $ \mathcal{B} $, we can calculate the matrix elements of the Green's function as
	\begin{align}
		\mel{\psi_i}{\hat{G}}{\psi_j} = \mel{\psi_i}{\left[ \left( \omega + \I \eta\right) \mathbb{I} - \ham \right]^{-1}}{\psi_j}
	\end{align}
	
	We now define the local spectral weight function as
	\begin{align}
		A_{i} \left( \omega \right) = -\frac{1}{\pi} \Im{ {\mel{\psi_i}{\hat{G}}{\psi_j}} }
	\end{align}
	It denotes the probability of finding the system within a given intervel of $ \omega $, provided it was initially in the state $ \ket{psi_i} $ and was allowed to evolve under the operation of the Hamiltonian.
	
	Following this, we now define the density of states as
	\begin{align}
		\rho \left( \omega \right) = \frac{1}{\mathcal{N}} \sum_{i = 1}^{\mathcal{N}} A_{i} \left( \omega \right)
	\end{align}
	where $ \mathcal{N} $ is the number of elements in $ \mathcal{B} $.
	
	In summary, we can calculate the density of states as
	\begin{align}
		\rho \left( \omega \right) = -\frac{1}{\mathcal{N \pi}} \Im{ \Tr( \left[ \left( \omega + \I \eta\right) \mathbb{I} - \ham \right]^{-1} ) }
	\end{align}
	We shall call this method of obtaining the Green's function and the related quantities as the \emph{direct inversion method}.
	
	Spectral weight function and density of states of importance as they can be directly measured in experiments through techniques such as ARPES (angle-resolved photoemission spectroscopy).
	
	\subsection*{Example}
		Consider a 1D crystal with N sites and N fermions. The system has $ \tfrac{N}{2} $ particles with spin $ +\tfrac{1}{2} $ and $ \tfrac{N}{2} $ with spin $ -\tfrac{1}{2} $. When two particles of opposite spins occupy the same site, an energy cost of $ U $ is added to the system per double-occupancy. The system evolves under the nearest-neighbor hopping tight-binding Hamiltonian,
		\begin{align}\label{eqn:tight-binding-hamil}
			\ham = -t \left(\sum_{\sigma = \uparrow, \downarrow} \sum_{\expval{i,j}} \hat{c}^{\dagger}_{i, \sigma} \hat{c}^{}_{j, \sigma} + \mathrm{h.c.} \right) + U \sum_{i}\hat{n}_{\uparrow} \hat{n}_{\downarrow}
		\end{align}
		
		Using the values $ N = 4 $, $ t = 1 $ and $ U = 4 $, we obtain the density of states shown in figure \ref{fig:4_4_0_4}.
		\begin{figure}[h!]
			\centering
			\includegraphics[width=1\linewidth]{4_4_0_4.pdf}
			\caption{DoS for the 4 site 4 particle half-illed Hubbard model, obtained by direct inversion.}
			\label{fig:4_4_0_4}
		\end{figure}

\section{Numerical issues with direct inversion}
	Although the Green's function method does not require costly diagonalizations, we instead incur the cost of inverting matrices. In the direct inversion method, we are required to invert a matrix the size of the Hamiltonian, whose dimension equals that of the basis size. For a 1D crystal with $ N $ sites, $ p_{\uparrow} $ spin-up particles and $ p_{\downarrow} $ spin-down particles, the size of the basis can be shown to be
	\begin{align}
		\mathrm{Size}\left( \mathcal{B} \right) =
		\frac{\left( N! \right)^{2} }{p_{\uparrow}!\,p_{\downarrow}!\,\left( N-p_{\uparrow}!\right) \left( N-p_{\downarrow}!\right) }
	\end{align}
	With $ p_{\uparrow} = p_{\downarrow} = \tfrac{N}{2} $, we get
	\begin{align}
		\mathrm{Size}\left( \mathcal{B} \right) =
		\frac{\left( N! \right)^{2} }{\left( \tfrac{N}{2}! \right)^{4}}
	\end{align}
	
	As $ N $ increases, this number increases exponentially. The cost of inverting an $ n \times n $ matrix can be shown to be of $ \order{n^3} $. Therefore, computers become increasingly slow at inverting these matrices as the system becomes bigger, and at one point, become incapable of even storing the Hamiltonian matrix in the RAM.
	
	\begin{figure}[h!]
		\centering
		\subfigure[Basis size with number of sites]{
			\includegraphics[width = 0.45\linewidth]{bsize.pdf}
			}
			\subfigure[The basis size in $ \log_{10} $ scale with  umbe rof sites]{
				\includegraphics[width = 0.45\linewidth]{bsizelog.pdf}
			}
		\caption{An ordinary and semilog plot showing how fast the basis size increases with incraesing number of sites. Even for relatively small sizes of $ N = 16 $, the basis size is of the order of $ 10^{8} $.}
	\end{figure}
	
	

	
	
	



