% Copyright 2004 by Till Tantau <tantau@users.sourceforge.net>.
%
% In principle, this file can be redistributed and/or modified under
% the terms of the GNU Public License, version 2.
%
% However, this file is supposed to be a template to be modified
% for your own needs. For this reason, if you use this file as a
% template and not specifically distribute it as part of a another
% package/program, I grant the extra permission to freely copy and
% modify this file as you see fit and even to delete this copyright
% notice. 

\documentclass{beamer}

\usepackage[english]{babel}
\usepackage{physics}
\usepackage{graphicx}
\usepackage{amsmath}
\usepackage{amsfonts}
\usepackage{amssymb}
\usepackage{subfigure}
\usepackage{caption}
\usepackage{hyperref}

% There are many different themes available for Beamer. A comprehensive
% list with examples is given here:
% http://deic.uab.es/~iblanes/beamer_gallery/index_by_theme.html
% You can uncomment the themes below if you would like to use a different
% one:
%\usetheme{AnnArbor}
%\usetheme{Antibes}
%\usetheme{Bergen}
%\usetheme{Berkeley}
%\usetheme{Berlin}
%\usetheme{Boadilla}
%\usetheme{boxes}
%\usetheme{CambridgeUS}
%\usetheme{Copenhagen}
%\usetheme{Darmstadt}
%\usetheme{default}
%\usetheme{Frankfurt}
%\usetheme{Goettingen}
%\usetheme{Hannover}
%\usetheme{Ilmenau}
%\usetheme{JuanLesPins}
%\usetheme{Luebeck}
\usetheme{Madrid}
%\usetheme{Malmoe}
%\usetheme{Marburg}
%\usetheme{Montpellier}
%\usetheme{PaloAlto}
%\usetheme{Pittsburgh}
%\usetheme{Rochester}
%\usetheme{Singapore}
%\usetheme{Szeged}
%\usetheme{Warsaw}

\newcommand{\I}{\mathrm{i}}
\newcommand{\ham}{\hat{\mathcal{H}}}

\renewcommand{\u}{\uparrow\ }
\renewcommand{\d}{\downarrow\ }
\newcommand{\ud}{\uparrow\downarrow}
\newcommand{\e}{\ \underline{\ \ }\ }
\newcommand{\p}{\ |\ }

\graphicspath{{./image/}}

\title{Generalization of Recursive Green's Functions for Many-body Systems}

% A subtitle is optional and this may be deleted
%\subtitle{Optional Subtitle}

\author{Amit Bikram Sanyal\inst{1}}
% - Give the names in the same order as the appear in the paper.
% - Use the \inst{?} command only if the authors have different
%   affiliation.

\institute[] % (optional, but mostly needed)
{
  \inst{1}%
  School of Physical Sciences\\
  National Institute of Science Education and Research
 }
% - Use the \inst command only if there are several affiliations.
% - Keep it simple, no one is interested in your street address.

\date{\today}
% - Either use conference name or its abbreviation.
% - Not really informative to the audience, more for people (including
%   yourself) who are reading the slides online

%\subject{Theoretical Computer Science}
% This is only inserted into the PDF information catalog. Can be left
% out. 

% If you have a file called "university-logo-filename.xxx", where xxx
% is a graphic format that can be processed by latex or pdflatex,
% resp., then you can add a logo as follows:

 \pgfdeclareimage[height=0.5cm]{university-logo}{image/logo1}
 \logo{\pgfuseimage{university-logo}}

% Delete this, if you do not want the table of contents to pop up at
% the beginning of each subsection:
\AtBeginSection[]
{
  \begin{frame}<beamer>{Outline}
    \tableofcontents[currentsection]
  \end{frame}
}

% Let's get started
\begin{document}

\begin{frame}
  \titlepage
\end{frame}

\begin{frame}{Outline}
  \tableofcontents[pausesections]
  % You might wish to add the option [pausesections]
\end{frame}

\section{Green's Function}
\subsection{Definition}
\begin{frame}{Definition}
	\begin{itemize}
		\item{
			Green's functions allow us to find the eigenvalues of an operator without resorting to costly exact diagonalization methods.
			}
		\item{ 
			The Green's function operator is defined as\,\cite{coleman_2015}
			\begin{align}\label{eqn:gf-def}
				\hat{G}\left(\omega\right) = \left[ \left( \omega + \I \eta\right) \mathbb{I} - \ham \right]^{-1}
			\end{align}
			}
		\item{
			For a non-interacting Hamiltonian, the Green's function operator is diagonal. In general, there are non-zero off-diagonal elements.
		}
	\end{itemize}
\end{frame}

\begin{frame}{Interpreting the eigenvalues}
	\begin{itemize}
		\item{
		For an eigenstate state $ \ket{\tilde{\psi_i}} $ of $ \ham $, the eigenvalue of $ \hat{G} $ is
		\begin{align}
		G\left(\omega\right) = \left[ \omega + \I \eta - \mathcal{E}_i \right]^{-1}
		\end{align}
		}
		\item {
		At the limit $ \eta \rightarrow 0 $, $ G(\omega) $ has a pole at $ \mathcal{E}_{i} $.
		}
		\item {
			Exact diagonalization methods such as \textit{Lanczos algorithm} forces us to calculate the entire spectrum at a time, while Green's function allows us to find the eigenvalues within any interval of $\omega$.
		}
	\end{itemize}
\end{frame}

\subsection{Calculation of physical observables}
	\begin{frame}{Calculation of physical observables}
		\begin{itemize}
		\item {
			Given a many body system, we can construct a many body basis consisting of the possible configurations of the system.
			\begin{align}
			\mathcal{B} = \{ \ket{\psi_1}, \ket{\psi_2}, \dots, \ket{\psi_{\mathcal{N}}} \}
			\end{align} 
			}
		\item {\pause
			The matrix corresponding to the Hamiltonian can be calculated element-wise.
			\begin{align}
				\mathcal{H}_{i,\,j} = \mel{\psi_i}{\ham}{\psi_j}
			\end{align}
			}
		\item {\pause
			The matrix elements of the Green's function are then given as
			 \begin{align}
			 \mel{\psi_i}{\hat{G}}{\psi_j} = \mel{\psi_i}{\left[ \left( \omega + \I \eta\right) \mathbb{I} - \ham \right]^{-1}}{\psi_j}
			 \end{align}
			}
	\end{itemize}
	\end{frame}
	
	\begin{frame}{Calculation of physical observables}
		\begin{itemize}
			\item {
				We now define the local spectral weight function as
				\begin{align}
				A_{i} \left( \omega \right) = -\frac{1}{\pi} \Im{ {\mel{\psi_i}{\hat{G}}{\psi_i}} }
				\end{align}
				}
			\item {\pause
				Following this, we now define the density of states as
				\begin{align}\label{eqn:dossum}
				\nonumber\rho \left( \omega \right) &= \frac{1}{\mathcal{N}} \sum_{i = 1}^{\mathcal{N}} A_{i} \left( \omega \right)\\
				&= -\frac{1}{\mathcal{N \pi}} \Im{ \Tr( \left[ \left( \omega + \I \eta\right) \mathbb{I} - \ham \right]^{-1} ) }
				\end{align}
				}
			\item {
				Density of states can be measured by experimental methods such as ARPES.
				}
		\end{itemize}
	\end{frame}
	
	\begin{frame}{Calculation of physical observables}
		\begin{itemize}
			\item{
				As an example, we consider a 1D crystal with $ N $ sites with $ \tfrac{N}{2} $ spin $ +\tfrac{1}{2} $ fermions and $ \tfrac{N}{2} $ spin $ -\tfrac{1}{2} $ fermions.
				}
			\item {
				The system evolves with the tight-binding Hamiltonian with on-site particle-particle interaction.
			\begin{align}\label{eqn:tight-binding-hamil}
			\ham = -t \left(\sum_{\sigma = \uparrow, \downarrow} \sum_{i=1}^{N-1} \hat{c}^{\dagger}_{i, \sigma} \hat{c}^{}_{i + 1, \sigma} + \mathrm{h.c.} \right) + U \sum_{i=1}^{N}\hat{n}_{i,\uparrow} \hat{n}_{i,\downarrow}
			\end{align} }
		\end{itemize}
	\end{frame}
	
	\begin{frame}{Calculation of physical observables}
		\begin{figure}[h!]
			\centering
			\includegraphics[width=0.7\linewidth]{6_6_0_4.pdf}
			\caption{DoS for the 6 site, 6 particle half-filled Hubbard model, obtained by direct inversion, with $ U=4 $.}
			\label{fig:6_6_0_4}
		\end{figure}
	\end{frame}

\subsection{Numerical cost analysis}
\begin{frame}{Numerical cost of direct inversion}
	\begin{itemize}
		\item{
			For a 1D crystal with $ N $ sites, $ p_{\uparrow} $ spin-up particles and $ p_{\downarrow} $ spin-down particles, and $ p_{\uparrow} = p_{\downarrow} = \tfrac{N}{2} $, the size of the basis can be shown to be
				\begin{align}
				\mathrm{Size}\left( \mathcal{B} \right) =
				\frac{\left( N! \right)^{2} }{\left( \tfrac{N}{2}! \right)^{4}}
				\end{align}
			}
		\item {\pause
			This number increases exponentially with $ N $.
			}
		\begin{figure}[h!]
			\centering
				\includegraphics[width = 0.3\linewidth]{bsize.pdf}\hspace{1cm}
				\includegraphics[width = 0.3\linewidth]{bsizelog.pdf}
			\caption{An ordinary and semilog plot showing how fast the basis size increases with increasing number of sites.}
		\end{figure}
	\end{itemize}
\end{frame}

\begin{frame}{Problems arising for large values of $ N $}
	\begin{itemize}
		\item Takes long time to calculate all configurations of the basis and all matrix elements.
		\item Difficult to store large $ N \cross N $ matrix in computer memory prior to inversion.
		\item Matrix inversion is an order $ \order{N^3} $ process, which becomes costlier in larger systems.
	\end{itemize}
\end{frame}

\section{Recursive Green's functions}
\subsection{One body recursive Green's functions}
	\begin{frame}{Idea behind recursive Green's functions}
		\begin{itemize}
			\item{
			Consider a one dimensional lattice with $ N $ sites.
			}
			\item {\pause
				We calculate the isolated Green's function of each site, which is the Green's function of a site in absence of hopping.
				\begin{align}
				g_{i} \left( \omega \right) = \left[ \omega + \I \eta - \epsilon_i \right]^{-1}
				\end{align}
				}
			\item{\pause
				We show that there exists a recursive relation for the full Green's function of each site, in terms of the isolated Green's function and the hopping constants between the sites.
				}
		\end{itemize}
	\end{frame}

\begin{frame}{Connecting isolated Green's functions}
	
\end{frame}

\begin{frame}{References}
	\bibliographystyle{plain}
	\bibliography{reflist.bib}
\end{frame}
\end{document}


