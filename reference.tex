\newcommand{\bib}{\bibitem}
%\newcommand{\bibitem}
\def\arpc{Ann. Rev. Phys. Chem.~~}
\def\irpc{Int. Rev. Phys. Chem.~~}
\def\jcp{J. Chem. Phys.~~}
\def\jacs{J. Am. Chem. Soc.~~}
\def\acp{Adv. Chem. Phys.~~}
\def\cp{Chem. Phys.~~}
\def\cpc{Comput. Phys. Commun.~~}
\def\cpr{Comput. Phys. Rep.~~}
\def\csr{Chem. Soc. Rev.~~}
\def\cpl{Chem. Phys. Lett.~~}
\def\cjc{Can. J. Chem.~~}
\def\cs{Curr. Sci.~~}
\def\acr{Acc. Chem. Res.~~}
\def\mph{Mol. Phys.~~}
\def\prl{Phys. Rev. Lett.~~}
\def\jpc{J. Phys. Chem.~~}
\def\joc{J. Org. Chem.~~}
\def\jms{J. Mol.  Structure~~}
\def\josab{J. Opt. Soc. Am. B~~}
\def\jao{J. Appl. Opt.~~}
\def\ijqc{Int. J. Quant. Chem.~~}
\def\sc{Science~~}
\def\tca{Theoret. Chim. Acta~~}
\def\pr{Phys. Rep.~~}
\def\ssr{Surf. Sci. Reports~~}
\def\prb{Phys. Rev. B~~}
\def\pra{Phys. Rev. A~~}
\def\prs{Proc. Roy. Soc.~~}
\def\pias{Proc. Indian Acad. Sci.~~}
\def\rpp{Rep. Prog. Phys.~~}
\def\pla{Phys. Letters A~~}
\def\fdcs{Faraday Discuss. Chem. Soc.~~}
\def\jcsft{J. Chem. Soc. Faraday Trans.~~}
\def\jrs{J. Raman Spectry.~~}
\def\rmp{Rev. Mod. Phys.~~}

\clearpage
\addcontentsline{toc}{chapter}{References}
\begin{thebibliography}{99}
%\bibitem{szabo} A. Szabo and N. S. Ostlund, Modern Quantum Chemistry, McGraw-Hill, New York, 1982.

\end{thebibliography}

\baselineskip 15pt
\setlength{\parskip}{10pt}

\noindent References are each numbered, ordered sequentially as they appear in the text, methods summary, tables, boxes, figure legends, etc. 

\noindent When cited in the text, reference numbers are {\bf superscript}, not in brackets unless they are likely to be confused with a superscript number.

\noindent {\large \bf Creating the Reference List} 

\noindent{\bf For journal articles,} list initials first for all authors, separated by a space: A. B. Opus, B. C. Hobbs. Do not use "and." Use {\em et al.} (italics) for more than five authors. Titles of cited articles should not be included (see samples). Journal titles are in italics; volume numbers follow, in boldface. Do not place a comma before the volume number or before any parentheses. You may give the full inclusive pages of the article. Journal years are in parentheses: (1996). End each listing with a period. Do not use {\em ibid.} or {\em op. cit.}.

\noindent {\bf For whole books, monographs, memos, or reports,} the style for author or editor names is as above; for edited books, insert "Ed.," or "Eds.," before the title. Italicize the book title and use initial caps. After the title, provide (in parentheses) the publisher name, publisher location, edition number (if any), and year. If these are unavailable, or if the work is unpublished, please provide all information needed for a reader to locate the work; this may include a URL or a Web or FTP address. For unpublished proceedings or symposia, supply the title of meeting, location, inclusive dates, and sponsoring organization. There is no need to supply the total page count. If the book is part of a series, indicate this after the title (e.g., vol. 23 of {\em Springer Series in Molecular Biology}).

\noindent {\bf For chapters in edited books,} the style is as above, except that "in" appears before the title, and the names of the editors appear after the title. After the information in parentheses, provide the complete page number range (or chapter number) of the cited material.

\noindent {\bf Style Examples} \\
\noindent {\bf Journals}
\begin{enumerate}
\item N. Tang, {\em Atmos. Environ.} {\bf 14}, 819-834 (1980). [one author]
\item W. R. Harvey, S. Nedergaard, {\em Proc. Natl. Acad. Sci. U.S.A.} {\bf 51}, 731-735 (1964). [two or more authors]
\item F. H. Chaffee, Jr., {\em Sci. Am.} {\bf 243}, 60-68 (November 1980).
\end{enumerate}

\noindent {\bf Books} 
\begin{enumerate}
\item M. Lister, Fundamentals of Operating Systems (Springer-Verlag, New York, ed. 3, 1984), pp. 7-11. [third edition]
\item J. B. Carroll, Ed., Language, Thought and Reality, Selected Writings of Benjamin Lee Whorf (MIT Press, Cambridge, MA, 1956).
\item R. Davis, J. King, in Machine Intelligence, E. Acock, D. Michie, Eds. (Wiley, New York, 1976), vol. 8, chap. 3. [use short form of publisher name, not "John Wiley \& Sons"]
\item D. Curtis et al., in Clinical Neurology of Development, B. Walters, Ed. (Oxford Univ. Press, New York, 1983), pp. 60-73. [use "Univ."]
\item Principles and Procedures for Evaluating the Toxicity of Household Substances (National Academy of Sciences, Washington, DC, 1977). [organization as author and publisher]
\end{enumerate}
\noindent {\bf Technical reports}
\begin{enumerate}
\item G. B. Shaw, "Practical uses of litmus paper in Möbius strips" (Tech. Rep. CUCS-29-82, Columbia Univ., New York, 1982).
\item F. Press, "A report on the computational needs for physics" (National Science Foundation, Washington, DC, 1981). [unpublished or access by title]
\item "Assessment of the carcinogenicity and mutagenicity of chemicals," WHO Tech. Rep. Ser. No. 556 (1974). [no author]
\item U.S. Environmental Protection Agency, The Environmental Protection Agency's White Paper on Bt Plant-Pesticide Resistance Management (EPA Publication 739-S-98-001, 1998; www.epa.gov/pesticides/biopesticides/whiteo\_bt.pdf). [the easiest access to this source is by Internet]
\end{enumerate}
\noindent {\bf Paper presented at a meeting (not published)}
\begin{enumerate}
\item M. Konishi, paper presented at the 14th Annual Meeting of the Society for Neuroscience, Anaheim, CA, 10 October 1984. [sponsoring organization should be mentioned if it is not part of the meeting name]
\end{enumerate}
\noindent {\bf Theses and personal communications}
\begin{enumerate}
\item B. Smith, thesis, Georgetown University (1973).
\item G. Reuter, personal communication. [Must be accompanied with a letter of permission and must not be used to support a central claim, result, or conclusion.]
\end{enumerate}

